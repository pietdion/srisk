\documentclass[authoryear]{elsarticle}
\usepackage{latexsym}
%\usepackage{rotate}
\usepackage{graphics}
\usepackage{amsmath}
\usepackage{amssymb}
\usepackage{comment}
\bibliographystyle{chicago}



\newcommand{\logit}{\mathrm{logit}}
\newcommand{\I}{\mathrm{I}}
\newcommand{\E}{\mathrm{E}}
\newcommand{\p}{\mathrm{P}}
\newcommand{\e}{\mathrm{e}}
\newcommand{\vecm}{\mathrm{vec}}
\newcommand{\kp}{\otimes}
\newcommand{\diag}{\mathrm{diag}}
\newcommand{\cov}{\mathrm{cov}}
\newcommand{\eps}{\epsilon}
\newcommand{\ep}{\varepsilon}
\newcommand{\obdots}{\ddots}    % change this later
\newcommand{\Ex}{{\cal E}}
\newcommand{\rat}{{\frac{c_{ij}}{c_{i,j-1}}}}
\newcommand{\rmu}{m}
\newcommand{\rsig}{\nu}
\newcommand{\fd}{\mu}
\newcommand{\tr}{\mathrm{tr}}
\newcommand{\cor}{\mathrm{cor}}
\newcommand{\bx}[1]{\ensuremath{\overline{#1}|}}
\newcommand{\an}[1]{\ensuremath{a_{\bx{#1}}}}

\newcommand{\bi}{\begin{itemize}}
\newcommand{\ei}{\end{itemize}}

\renewcommand{\i}{\item}
\newcommand{\sr}{\ensuremath{\mathrm{SRISK}}}
\newcommand{\cs}{\ensuremath{\mathrm{CS}}}
\newcommand{\cri}{\ensuremath{\mathrm{Crisis}}}
\newcommand{\var}{\ensuremath{\mathrm{VaR}}}
\newcommand{\covar}{\ensuremath{\mathrm{CoVaR}}}
\newcommand{\med}{\ensuremath{\mathrm{m}}}
\newcommand{\de}{\mathrm{d}}
\renewcommand{\v}{\ensuremath{\mathrm{v}_q}}
\newcommand{\m}{\ensuremath{\mathrm{m}}}
\newcommand{\tvar}{\ensuremath{\mathrm{TVaR}}}



\newcommand{\eref}[1]{(\ref{#1})}
\newcommand{\fref}[1]{Figure \ref{#1}}
\newcommand{\sref}[1]{\S\ref{#1}}
\newcommand{\tref}[1]{Table \ref{#1}}
\newcommand{\aref}[1]{Appendix \ref{#1}}




\newcommand{\cq}{\ , \qquad}
\renewcommand{\P}{\mathrm{P}}
\newcommand{\Q}{\mathrm{Q}}

\newcommand{\Vx}{{\cal V}}
\newcommand{\be}[1]{\begin{equation}\label{#1}}
\newcommand{\ee}{\end{equation}}




\begin{document}


\section{Bivariate case}

Univariate stressing leads to the stressed put expectation
$$
\E\{\phi(r_m)p_{it}\} = \int  f(r_m)\phi(r_m)\E(p_{it}|r_m) \de r_m = \int f_\phi(r_m) \E(p_{it}|r_m) \de r_m
$$
where $f$ is the original density of a market factor $r_m$ and $f_\phi=f\times \phi$ is the assumed density. Note $\phi$ may act on $r_m$ or its percentile rank $u_m$. The intent of $\phi$ is the enlarge the likelihood of ``bad" outcomes of $r_m$.

Bivariate stressing yields
$$
\E\{\phi(r_m,r_n)p_{it}\} = \int \int f(r_m,r_n)\phi(r_m,r_n) \E(p_{it}|r_m, r_n) \de r_m \de r_n
$$
$$
= \int \int f_\phi(r_m,r_n) \E(p_{it}|r_m, r_n) \de r_m \de r_n
$$
where $r_n$ is another market factor and $f$ is now the joint density of $(r_m,r_n)$. The assumed joint density is $f_\phi=f\times \phi$, which again enlarges the joint likelihood of ``bad" outcomes of $(r_m,r_n)$. For example $f_\phi$ may have stronger tail dependence than $f$, which can be achieved by either selecting a copula which has stronger tail dependence, or setting up $\phi$ as $\phi(u_mu_n)$ where $\phi$ is decreasing. Marginal distributions of $r_m$ and $r_n$ may be altered or unchanged. If unchanged then
$$
f(r_n) = \int f(r_m,r_n) \de r_m = \int f_\phi(r_m,r_n) \de r_m
$$
or 
$$
1 = \int f(r_m)c(u_m,u_n) \de r_m = \int f(r_m)\phi(r_m,r_n) c(u_m,u_n) \de r_m
$$
since $f(r_m,r_n)=f(r_m)f(r_n)c(u_m,u_n)$.


\subsection{Copula stressors}

\newcommand{\Es}{\E_\phi}

Stressing can be based on a vector of variables $x$ with distribution $F(x)=u$.   Given a copula $\Phi(u)$ with density $\phi(u)$ consider
the stressed expectation 
\be{copula}
\Es(p_{it}) \equiv \int \E(p_{it}|x)  \de\Phi\{F(x)\}  =  \E\{\phi(u)\E(p_{it}|x)\}  \ ,
\ee
The  copula $\Phi(u)$ is designed to magnify  stressful situations such as  where all components of $x$ are abnormally low.
If $\Phi(u)=u$ then $\phi(u)=1$ and $\Es(p_{it})=\E(p_{it})$.  

For example $\Phi\{F(x)\}$ may have stronger tail dependence than $F$.   In the extreme  $\Phi(u)=\min_i(u_i)$ making all  variables in $x$  comonotonic.   The copula $\Phi(u)$  leaves marginals intact and acts as a stressor on the copula  $c(u)$  corresponding to $F$ since
$$
\de\Phi\{F(x)\} = \phi(u)c(u)\prod_if_i(x_i)\ ,
$$
where the $f_i$ are the  marginal densities of $F$.   Further marginal stress is induced by allowing $\Phi(u)$ to have non--uniform marginals in which case
$$
\de \Phi(u) = \phi(u)\prod_i\phi_i(u_i)\de u\ ,
$$
where $\phi(u)$ has, as before, uniform marginals and the $\phi_i(u_i)\ge 0$ induce non--uniformity in each of the marginals subject to $\E\{\phi_i(u_i)\} = 1$.  

   
\end{document}
