\documentclass[authoryear]{elsarticle}
\usepackage{latexsym}
%\usepackage{rotate}
\usepackage{graphics}
\usepackage{amsmath}
\usepackage{amssymb}
\usepackage{comment}
\bibliographystyle{chicago}



\newcommand{\logit}{\mathrm{logit}}
\newcommand{\I}{\mathrm{I}}
\newcommand{\E}{\mathrm{E}}
\newcommand{\p}{\mathrm{P}}
\newcommand{\e}{\mathrm{e}}
\newcommand{\vecm}{\mathrm{vec}}
\newcommand{\kp}{\otimes}
\newcommand{\diag}{\mathrm{diag}}
\newcommand{\cov}{\mathrm{cov}}
\newcommand{\eps}{\epsilon}
\newcommand{\ep}{\varepsilon}
\newcommand{\obdots}{\ddots}    % change this later
\newcommand{\Ex}{{\cal E}}
\newcommand{\rat}{{\frac{c_{ij}}{c_{i,j-1}}}}
\newcommand{\rmu}{m}
\newcommand{\rsig}{\nu}
\newcommand{\fd}{\mu}
\newcommand{\tr}{\mathrm{tr}}
\newcommand{\cor}{\mathrm{cor}}
\newcommand{\bx}[1]{\ensuremath{\overline{#1}|}}
\newcommand{\an}[1]{\ensuremath{a_{\bx{#1}}}}

\newcommand{\bi}{\begin{itemize}}
\newcommand{\ei}{\end{itemize}}

\renewcommand{\i}{\item}
\newcommand{\sr}{\ensuremath{\mathrm{SRISK}}}
\newcommand{\cs}{\ensuremath{\mathrm{CS}}}
\newcommand{\cri}{\ensuremath{\mathrm{Crisis}}}
\newcommand{\var}{\ensuremath{\mathrm{VaR}}}
\newcommand{\covar}{\ensuremath{\mathrm{CoVaR}}}
\newcommand{\med}{\ensuremath{\mathrm{m}}}
\newcommand{\de}{\mathrm{d}}
\renewcommand{\v}{\ensuremath{\mathrm{v}_q}}
\newcommand{\m}{\ensuremath{\mathrm{m}}}
\newcommand{\tvar}{\ensuremath{\mathrm{TVaR}}}



\newcommand{\eref}[1]{(\ref{#1})}
\newcommand{\fref}[1]{Figure \ref{#1}}
\newcommand{\sref}[1]{\S\ref{#1}}
\newcommand{\tref}[1]{Table \ref{#1}}
\newcommand{\aref}[1]{Appendix \ref{#1}}




\newcommand{\cq}{\ , \qquad}
\renewcommand{\P}{\mathrm{P}}
\newcommand{\Q}{\mathrm{Q}}

\newcommand{\Vx}{{\cal V}}
\newcommand{\be}[1]{\begin{equation}\label{#1}}
\newcommand{\ee}{\end{equation}}




\begin{document}


\section{Weihao's observations}

\bi

\i How to standardise $\cov(p_{it},\phi)$ -- still thinking but inclined to standardise by $\sigma_\phi$ instead of $\sigma_\phi^2$.

pdj -- I agree.  Then one is inclined  to write
$$
p_{it} \approx \mu_{it} + \beta_{it}\frac{\phi-1}{\sigma_\phi}\ ,
$$
and hence $\beta_{it}$ is the response to a one--unit increase in the standardised stress factor $(\phi-1)/\sigma_\phi$.  Hence to my mind it argues the stress factor $(\phi-1)/\sigma_\phi$ is core and needs concrete characterisation.



\i Capital shortfall at time $i+h$ is the difference between debt and admissible assets
$$
d_{it}-(1-k)(w_{i,t+h}+d_{it})  \cq w_{i,t+h}=w_{it} \e^{\nu_{it}}
$$
where $k$ is the non-admissible factor and $\nu_{it}$ is the return on equity. Capital shortfall at time $i+h$ can be rewritten as
$$
d_{it} - (1-k_{i,t+h})(w_{it}+d_{it})  \cq k_{i,t+h} \equiv 1-(1-k)\frac{w_{it}\e^{\nu_{it}}+d_{it}}{w_{it}+d_{it}}
$$
where assets are constant, similar to debt. However the non-admissible or impairment factor $k_{i,t+h}$ varies inversely with $\nu_{it}$. One can either forecast capital shortfall by modeling $\nu_{it}$ or $k_{i,t+h}$, and this paper adopts the former due to available data on equity returns.

pdj -- Another way to look at this is to write
$$
\frac{1-k_{i,t+h}}{1-k} = \left(1-\frac{d_{it}}{a_{it}}\right)\e^{\nu_{it}}+\frac{d_{it}}{a_{it}}
$$
Can this be exploited? -- note debts to assets is the odds of debt to equity


\i Artzner asserts that any coherent risk measure is the supremum of the expected capital shortfall in a collection of generalised scenarios or probability measures $\mathcal{P}$ on states of the world, i.e.
$$
\sup \left\{ \E_\mathbb{P}(p_{it}) | \mathbb{P} \in \mathcal{P} \right\}
$$
pdj-- this seems a useful background result allowing us to argue the generality of our approach.  Not sure how to best write up/connect to our presentation.


\i $\tilde{\E}(p_{it}) = \E(p_{it}) + \cov\{\E(p_{it}|\phi),\phi\}$

pdj-- The implication here is that one can take each given scenario (ie possible outcome of $\phi$) and just consider how expected put prices for each scenario covary with scenario:
$$
\beta_{it} = \cov\{p_{it}(\phi^*),\phi^*\}\cq \phi^* \equiv \frac{\phi-1}{\sigma_\phi}\cq p_{it}(\phi^*)\equiv\E(p_{it}|\phi^*)\approx \mu_{it}+\beta_{it}\phi^*
$$
This seems useful to me.

\i Write the put payout as
$$
p_{it}=k\left|1-\e^{\nu_{it}^*}\right|^+ = k \int_0^\infty (1-\e^{\nu_{it}^*} > s) \de s
=  k \int_0^\infty (\nu_{it}^* < \ln(1-s)) \de s
$$
$$
= k \int_{-\infty}^0 (\nu_{it}^* < s) \e^s \de s
$$
and hence the put covariance is
$$
\cov(p_{it},\phi) = \cov\left\{ k \int_{-\infty}^0 (\nu_{it}^* < s) \e^s \de s , \phi \right\}
= k \int_{-\infty}^0  \cov\left\{  (\nu_{it}^* < s) , \phi \right\} \e^s \de s \;.
$$
The covariance in the final integral is between a capital shortfall scenario $(\nu_{it}^* < s)$ and the stress factor. 

pdj -- can you use the expected stress factor trick here?  Still thinking about the implications of above. 
\ei
\end{document}
