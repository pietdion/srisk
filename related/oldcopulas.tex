\input{preamble}

\section*{Copulas}
A copula is a probability distribution $C$ on the unit hypercube with uniform marginals.  Given a vector $F_*$ of marginal distributions\footnote{Formally $F_*(y)$ has components $F_i(y_i)\equiv F(y_i^\infty)$ where $y_i^\infty\equiv(\infty,\ldots,y_i,\ldots\infty)$} and copula $C$ define $F_C$ as follows\footnote{The composition $G\circ H$ of two functions $G$ and $H$ is defined as $(G\circ H)(x)\equiv G\{H(x)\}$ for all $x$.} \begin{equation}\label{Fx}
    F_C\equiv C\circ F_*\ .
\end{equation}
This is  a joint distribution\footnote{Since $(C\circ F_*)_*(y)$ has components $(C\circ F_*)(y_i^\infty)=C\{F_*(y_i^\infty)\}=C_i\{F_i(y_i)\}=F_i(y_i)$} with the given marginals $F_*$  Conversely given a joint distribution $F$ with marginals $F_*$ and $F_*^-$ as the componentwise inverse to $F_*$ define
\begin{equation}\label{Cu}
    C_F\equiv F\circ F_*^-\ .
\end{equation}
This is a copula, the copula of $F$.   Further
$$
F_{C_F}= C_F\circ F_*= F\circ F_*^-\circ F_*=F\ , \qquad C_{F_C}=F_C\circ (F_C)_*^-=C\circ F_*\circ  F_*^-=C\ .
$$
Hence (Sklar's theorem) $F_C=F$ iff $C_F=C$.  

\subsubsection{Copula properties} 
\begin{itemize}
    \item  Each component of $F_*$ is arrived at by setting all arguments, other than the current, to $\infty$.  Further
     $F_*^-$ depend only on the corresponding component of the argument.
     \item If $y\sim C\circ F_*$ then $y=F_*^-u$ where $u\sim C$.
    \item  If $H_*$ is the componentwise  vector of univariate monotonic functions then
$$
C_{F\circ H_*^-}\equiv F\circ H_*^- \circ(F\circ H_*^-)_*^-=F \circ F_*^-\equiv C_F\ .
$$
Hence copulas are invariant to monotonic deformations along each axis. 
    \item The density associated with $C_F=F\circ F_*^-$ is
    $$
    \frac{f\circ F_*^-}{\Pi (f_*\circ F_*^-)}\ ,
    $$
    where $f$ is the density associated with $F$ and $f_*$ is the vector of marginal distributions.  If $f=\Pi f_*$ (ie independence) then $f\circ F_*^-=\Pi (f_*\circ F_*^-)$ and the density is identically 1.
    \item The Gaussian copula with correlation matrix $R$ is the copula $C_{\Phi^R}$  where $\Phi^R$ is the Gaussian distribution with mean 0 and covariance matrix $R$.   Since $R$ is a correlation matrix, $\Phi^R_*=\Phi_*$, a vector of standard normal distributions, and $C_{\Phi^R}=\Phi^R\circ \Phi_*^-$.   The meta--Gaussian model is
\begin{equation}\label{metaG}
y = F_*^-u\ , \qquad u\sim\Phi^R\circ\Phi_*^-\qquad\Leftrightarrow\qquad y =(F_*^-\circ\Phi_*^-)\sqrt{R}\eps\ , \qquad \eps\sim \Phi^I\ .
\end{equation}
Typically $R$ is estimated using the average cross product matrix of  the normits  $(\Phi_*^-\circ F_*) y$.
   
\item The independence and perfectly dependent copulas is defined as $\Pi\equiv\Pi\ 1_*$ and $\vee\equiv\min$ respectively, where $1_*$ indicates the vector of identity functions.  These arise as follows:
$$
C_{\Pi F_*}\equiv(\Pi F_*)\circ F_*^-= \Pi\ 1_*\ ,\qquad C_{\vee}=\ .
$$
The perfectly dependent copula implies a single source of noise: 
$y=F_*^-1u$ with $u\sim U$
where $1$ is a vector of  identity functions of one variable.
\end{itemize}

\section{Mortality smoothing}

There are two ways of applying copulas to mortality smoothing:
\begin{enumerate}
    \item The copula $C$ models correlation across age and $u\sim C$ is a ``time" draw. Age specific  behavior across time defines the components of $F_*$.  The age profiles are generated according to $y=F_*^-u$ with $u\sim C$ and different draws correspond to different times. Marginals in $F_*$ corresponding to each age  are estimated from the ergodic average
$
\hat F_*(s) =\Ex[y<s]
$, where $\Ex$ denotes the ergodic average.
Special cases are:
    \begin{enumerate}
        \item Time profiles do not vary with age implying each marginal in $F_*$ is the same.  Hence $F_*=1\odot h$ and $h$ would be estimated as $\hat h(s)\equiv\Ex\{\hat F_*(s)\}=\Ex\{\Ex[y<s]\}$ where the outer expectation is the ensemble average.
        \item \label{pd} Perfect dependence across age implies all the time series are driven by a single source of noise: $y= F_*^-1u$ where $u\sim U$.
        \item \label{ind} Independence across age implies $ y= F_*^-u$  where $u\sim \Pi$.  This model invokes the unlikely assumption that each age moves independently over time.
        \item   Intermediate cases are  $ y=F_*^-u$ with $u\sim C$.   With the meta--Gaussian model, the correlation matrix $R$ models the correlation between ages and generally the correlation between close ages is  higher.   An extreme case is where the correlation between ages $i$ and $j$ is $\rho^{|i-j|}$ for $0\le\rho\le 1$.  The case $\rho=1$ is equivalent to \ref{pd} while $\rho=0$ to \ref{ind} above.
    \end{enumerate}        
    \item The copula $C$ models serial correlation and $u\sim C$ is an ``age" draw.  Time specific mortality curves are the components in $F_*$.   The time series profiles $y$ of mortality are generated  according to $y=F_*^-u$ where $u\sim C$ and different draws correspond to different ages.  Marginals in $F_*$ corresponding to each $t$ are estimated from the ensemble (age profile) data at time $t$, $\hat F_*(k)\equiv\Ex[y<k]$ where the expectation is empirical average is across the (age) different draws.
Special cases are:
\begin{enumerate}
    \item Age profiles do not vary with time implying each marginal in  $F_*$ is the same.  Hence $F_*=1\odot h$ where $h$ is univariate and $(1\odot h) y$ applies $h$ componentwise.  Thus $y=(1\odot h)^-u=(1\odot h^-)u$.  Map $h$ is estimated from the ergodic average of the empirical ensemble (age profile) distributions
$
\hat h(k)\equiv \Ex\{\hat F_*(k)\}=\Ex\{\Ex[y<k]\}
$ where the outer $\Ex$ works across the components of the vector.
    \item Perfect serial dependence implies $y= F_*^-1u$ where $u\sim U$. Combined with a single age profile yields $y=(1\odot h^-)1u=1h^-u$ with $u\sim U$.  To estimate $h$ is $\hat h(k)=\Ex\{\Ex[y<k]\}$.   
    \item Matrix $R$ in the meta--Gaussian model is a time series covariance matrix.  The correlation matrix is estimated from the $z\equiv(\Phi_*^-\circ F_*)y$ as
$
\hat R = \Ex(zz')
$.
Note $z\sim \Phi^R$ and $(\Phi^R)_*=\Phi_*$.  In the fixed mortality curve situation $z=\{\Phi_*^-\circ(1\odot h)\}y$.  Thus each component of $y$ is transformed by $h$.  And $h$ would be estimated as $\hat h(k)\equiv\Ex\{\Ex[y<k]\}$.

\item    With the meta--Gaussian model the stationary AR(1) case  is where entry $(t,s)$ of $R$ is $\rho^{|t-s|}$ where $0\le\rho\le 1$.  This meta--Gaussian model combined with a fixed mortality curve implies 
$$
y=\{(1\odot h^-)\circ \Phi_*^-\}\sqrt{R}\eps\ , \qquad \eps\sim \Phi^I\ .
$$
Given observed time profiles $y$ for each age, the matrix $R$ is estimated by first transforming to normits 
$
z=\{\Phi_*\circ(1\odot h)\}y
$ and $\hat R = \Ex(zz')$.  The estimate of $\rho$ is $\Ex(r_{t,t-1})$ where the average is over $t$. 
\end{enumerate}     
\end{enumerate}    



\section{Usefulness of copulas in applied work}

 The above indicates there are two ways of thinking about copulas:
\begin{enumerate}
    \item As a way of generating a joint from given marginals.
    \item As a way of writing the joint in terms of its marginals.
\end{enumerate}    
In applied work the first seems to be the most important.   In other words given marginals can be ``moulded" into many different joint distributions through choice of an appropriate copula.




 is first adjusted by the use of explanatory variables to indicate it is coming from different populations.

\subsubsection{Normit modelling}

Suppose the joint distribution of $\psi(x)$ and $\psi(y)$ is bivariate normal with correlation $\rho$.   Then since $\psi(x)$ and $\psi(y)$ are standard normal  
$\rho=\E\{\psi(y)\psi(x)\}$.

For example using claim size and settlement delay as an example then the correlation between the normits is 0.68.   Hence
$$
\psi(\hat y)=0.68\psi(x)\qquad\Rightarrow\qquad  F(\hat y) =\Phi\{0.68\psi(x)\}\ .
$$
That is, given $x$  the prediction $\hat y$ has predicted quantile $\Phi\{0.68\psi(x)\}$.  The actual prediction $\hat y$ is determined by going into the distribution of $y$ and picking out the value corresponding to the predicted quantile.

\subsubsection{Simulating from Gaussian copula}

To simulate we just generate standard normals with a given correlation matrix.   That is suppose the correlation is $\rho$ then given two standard normals $z_1$ and $z_2$ we form
$$\left(%
\begin{array}{c}
  \psi(x) \\
  \psi(y) \\
\end{array}%
\right)=
\left(%
\begin{array}{cc}
  1 & \rho \\
  \rho & 1 \\
\end{array}%
\right)^{1/2} \left(%
\begin{array}{c}
  z_1 \\
  z_2 \\
\end{array}%
\right)
$$ 
We now form $\Phi\{\psi(x)\}$ and $\Phi\{\psi(y)\}$

\subsubsection{Runoff triangles}

Think of each row as a distribution $F_i(j)$ where $i$ is the accident year and $j$ the development year.  We can convert this to normits by taking $\psi_i(j)\equiv\Phi^{-1}\{F_i(j)\}$.  We now study the correlations across the rows.  Trouble is we don't know the quantiles.

Note $\psi_1(n)=$

Now take the slope
$$
\psi_i(j) = a_i+b_ij
$$


\subsubsection{Joint distribution under Gaussian copulas}




We want to find the joint distribution of $(x,y)$.  The general result about this sort of thing is as follows.   If $(x^*,y^*)$ has density $f(x^*,y^*)$ and $(x^*,y^*)=\phi(x,y)$ is one--to--one, then the joint density of $(x,y)$ is
$$
f\{\phi(x,y)\}\left|\frac{\partial \phi(x,y)}{\partial (x,y)}\right|=f\{\psi(x),\psi(y)\}\dot\psi(x)\dot\psi(y)\ ,
$$
Now
$$
\dot\psi(x) = \frac{\partial x^*}{\partial x^\dag}\frac{\partial x^\dag}{\partial x}= \left(\frac{\partial x^\dag}{\partial x^*}\right)^{-1}\frac{\partial x^\dag}{\partial x}=\left\{\frac{\partial \Phi(x^*)}{\partial x^*}\right\}^{-1}\frac{\partial x^\dag}{\partial x}=\sqrt{2\pi}\exp\left\{\frac{\psi^2(x)}{2}\right\}f(x)
$$
Hence if $f\{\psi(x),\psi(y)\}$ is bivariate standard normal with correlation $\rho$ then
the joint density  of $(x,y)$ is
$$
f(x,y)=\frac{1}{1-\rho^2}\exp\left\{\frac{\psi^2(x)+\psi^2(y)}{2}-\frac{\psi^2(x)-2\rho\psi(x)\psi(y)+\psi^2(y)}{2(1-\rho^2)}\right\} f(x) f(y)
$$$$
=\Phi\{\psi(x),\psi(y)\}f(x)f(y)
$$
where $f(x)$ and $f(y)$ are the marginal densities of $x$ and $y$ and
$$
\Psi_\rho(u,v)= \frac{1}{1-\rho^2}\exp\left\{-\frac{\rho^2(u^2+v^2)-2\rho uv}{2(1-\rho^2)}\right\}\ .
$$

Note the marginals are $f(x)$ and $f(y)$ for any choice of $\rho$ and if $\rho=0$ then the joint density is $f(x)f(y)$, as expected. Further since $f(x,y)=f(x)f(y|x)$ it follows
$$
f(y|x)=\Psi_\rho\left\{\psi(x),\psi(y)\right\}f(y)\qquad\Rightarrow\qquad \frac{f(y|x)}{f(x|y)}=\frac{f(y)}{f(x)}\ .
$$ 

\subsubsection{Sampling from the density}

Differentiating  $\Psi_\rho(u,v)$ with respect to $v$ yields 
$$
\Psi_\rho(u,v)\frac{\rho(\rho v- u)}{(1-\rho^2)}\ .
$$ 
Thus the maximum occurs at $v=u/\rho$ in which case 
$$
\Psi_\rho(u,v)=\frac{1}{1-\rho^2}\exp\left\{\frac{-u^2(1+\rho^2)+2u^2}{2(1-\rho^2)}\right\}=\frac{1}{1-\rho^2}\exp\left(\frac{u^2}{2}\right)\ .
$$
Thus 
$$
f(y|x)\le \frac{1}{1-\rho^2}\exp\left\{\frac{\psi^2(x)}{2}\right\}f(y) \ .
$$
Hence the ratio of the left to right hand side is
$$
\frac{\Psi_\rho\left\{\psi(x),\psi(y)\right\}}{\frac{1}{1-\rho^2}\exp\left\{\frac{\psi^2(x)}{2}\right\}}=\exp\left\{-\frac{\psi^2(x)}{2}-\frac{\rho^2\psi^2(x)+\rho^2\psi^2(y)-2\rho \psi(x)\psi(y)}{2(1-\rho^2)}\right\}  
= \exp\left[-\frac{\{\psi(x)-\rho\psi(y)\}^2}{2(1-\rho^2)}\right]
$$
Thus a rejection algorithm for generating samples from the joint is
\begin{enumerate}
    \item  $x\sim f(x)$
    \item Repeat $y\sim f(y)$ and $u \sim U(0,1)$ until $$
    \ln(u) < \frac{-\{\psi(x)-\rho\psi(y)\}^2}{2(1-\rho^2)}\ .
    $$
\end{enumerate}
The resulting draw $(x,y)$ is a draw from $f(x,y)$ generated using the Gaussian copula.   Note that if $\rho\approx 1$ then $u$ must be very small to get an accepted $y$.  If $\rho=0$ then $y$ is accepted if $\ln(u)<-\psi^2(x)/2$.  Since this latter condition makes no reference to $y$, all draws $y$ can be accepted.

Since $\psi(x)$ and $\psi(y)$ are standard normal we can just generate lots of these and only accept those meeting the condition in 2.

\section{What copula is appropriate?}

Suppose we have data vectors $x_1,\ldots,x_n$.  Corresponding to the measurements calculate the quantiles $u_1,\ldots, u_n$ or, in the case of the Gaussian copula, the normits, $z_1,\ldots,z_n$. We study the behavior of the  

$F_*(x)$ denotes the vector of fitted marginals.  Then convert the data to quantiles

\section{Archimedean copulas}

These are of the form
$$
C(u,v)=A^-\{A(u)+A(v)\}
$$


\section{Reserving}
Suppose the correlation between $\psi(y)$ and $\psi(x)$ is $\rho$ and suppose the two transformed variables are jointly normal.   We want to find $c_\alpha$ such that
$$
P(x+y<c_\alpha)=\alpha\ .
$$
One approach is
$$
\int_x\int_y \Psi_\rho\left\{\psi(x),\psi(y)\right\}f(x)f(y)\ .
$$

Lets see if we can find the joint distribution of $(x,y)$ given the joint distribution of $\{\psi(x),\psi(y)\}$.

\section{Calculations} 
In this case $\phi(x,y)=\{\psi(x),\psi(y)\}$ and  the determinental term is $\psi(x)\psi(y)$.
is the density of 
in this case $z=(x^*,y^*)'=\{\psi(x),\psi(y)\}'$ and $h(z)=(x,y)'=\{\psi^{-1}(x^*),\psi^{-1}(y^*)\}$.   Thus $h^{-1}(z)=$
$$
\left|\frac{\partial h^{-1}(z)}{\partial z}\right| = \left|\right|
$$
In general terms define $\Omega_c=\{(x,y):x+y<c\}$


 and
$$
\E\{\psi(y)|x\}=\rho\psi(x)\ , \qquad Var\{\}
$$
$$
\E(y|x)=\tilde F_y^{-1}[\Phi\{\rho\psi(x)\}]\ , \qquad Var\{\}
$$
But the left hand side is approx
$$
\E\{\phi(\mu_{y|x})+\dot\phi(\mu_{y|x})(y-\mu_{y|x})|x\} = \phi(\mu)+\dot\phi(\mu)\{\E(y|x)-\mu\}\ .
$$
This implies
$$
\E(y|x)\approx\mu+\frac{\rho\phi(x)-\phi(\mu)}{\dot\phi(\mu)}\ .
$$
Thus to calculate the conditional expectation of 



To calculate this redefine $y$ to be two dimensional and with $y^*=\phi(y)$


for any $0<\alpha<1$ we can find $c_\alpha$ such that
$$
P\{\phi(y)+\phi(x)<c_\alpha\}=\alpha\ .
$$
The issue is to compute the region $\{(y,x): \phi(y)+\phi(x)<c_\alpha\}$.

One way of computing the region is:
\begin{itemize}
    \item Pick random $(y^*,x^*)$ and check if $y^*+x^*<c_\alpha$
    \item If in the region then compute $y=\phi^{-1}(y^*)$ and $x=\phi^{-1}(x^*)$ 
    \item 
\end{itemize}     

We now try and model the relationship between $y^*$ and explanatory variables $x^*$. Consider the normal model 
$$
y^* \sim N({x^*}'\beta,\sigma^2)\ .
$$
We fit the model and obtain predictions $\hat y^*$ and in turn $\hat y^\dag=\Phi(\hat y^*)$.  This is a prediction of the percentile of $y$ from the percentiles of $x$.  Note that the fitted vector $\hat\beta$ will be a vector of partial correlations.

Suppose now $y$ is a vector of variables (not observations).  We want the distribution of $\iota'y$ representing the total liability.  We can measure/infer the distribution of normalized scores $y^*$.   In particular $y^*\sim N(0,\Sigma)$ where $\Sigma$ is a correlation matrix since it has all 1's on the diagonal.  Thus the distribution of $\iota'y^*$ is $N(0,\iota'\Sigma\iota)$.   The $\alpha$ percentile is given by $z_\alpha\sqrt{\iota'\Sigma\iota}$ where $z_\alpha$ cuts out the required probability in a standard normal:
$$
P(\iota'y^*<z_\alpha\sqrt{\iota'\Sigma\iota})=\alpha\ .
$$

How do we convert this to a probability statement on the actual total $\iota'y$?  That is we want the set  
$$
\Omega_\alpha = \left\{y : \iota'y^*<z_\alpha\sqrt{\iota'\Sigma\iota}\right\}\ .
$$
Note that the set of $y^*$ satisfying the region is not a quadrant and the normal is centred on zero with unit variances.
One way of doing this is to see where each $y^*$ comes from.   

For example in the two dimensional case this is 

 In moving from the original data $y_i$ to $y_i^\dag$ and $y_i^*$ we lose all information about location and scale.  Lets have a separate model for that.

Since  $y^\dag=\Phi(y^*)=\pi(y)$.  Hence the density of $\pi(y)$ is uniform with mean 0.5.
$$
\frac{\partial \Phi^{-1}(y^*)}{\partial y^*} \exp\left\{-\frac{(y^*-{x^*}'\beta)^2}{2\sigma^2}\right\}= 
$$
 

For given data we can then consider a normal model 
$$
y^* \sim N({x^*}'\beta,\sigma^2)\ .
$$
But $\Phi(y^*)=\pi(y)$.  Hence the density of $\pi(y)$ is uniform with mean 0.5.
$$
\frac{\partial \Phi^{-1}(y^*)}{\partial y^*} \exp\left\{-\frac{(y^*-{x^*}'\beta)^2}{2\sigma^2}\right\}= 
$$

Thus the log--density of $y^*$ is proportional to 
$$
\frac{y*}{}
$$

The copula approach is to design a distribution on the unit hypercube $C(u)$ with uniform marginals.  Consider first 
the bivariate case.  Then $C(u,v)$

An alternative is to use the $\phi(y_i)$.  That is we just give them a covariance matrix.

Since the former are probabilities it is convenient to turn them into logits.



  Similarly let $z(y)$ be the ``zscore" of $y$, meaning we calculate the uniform then find $z$ such that
$$
z(y)=\Phi^{-1}\{u(y)\}\ ,\qquad z(y_i)=\Phi^{-1}\left\{\frac{1}{n} \sum_{j=1}^n[y_j<y_i]\right\}\ .
$$
This is different from the usual z--score calculated as
$$
\frac{y-\bar y}{s_y}
$$


For example suppose the relationship between the percentiles of two variables can be adequately described by 
$$
u(y) \approx \beta_0 + \beta_1 u(x)\ ,
$$
or from a glim perspective
$$
\E\{U(y)\}=\beta_0+\beta_1 U(x) \ , \qquad Var\{U(y)\}=\gamma_0+\gamma_1 U(x)+\gamma_2 U^2(x)\ .
$$
\section{Example copulas}

\subsubsection{Independence copula.}   Here $C(u)=\prod_ju_j$.   This would imply there is no correlation between the percentiles.  

\section{Estimating copulas}

Copulas can be estimated by first using the percentile transform each variable.   Then plot these percentiles against each 
other.   What is the joint distribution of the percentiles.   One approach is  
The way to estimate copulas is to 
In many cases we observe the variables and have some information on the joint distribution.

Different copulas induce different dependence structure.   

\end{document} 