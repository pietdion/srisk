\documentclass[authoryear]{elsarticle}
\usepackage{latexsym}
%\usepackage{rotate}
\usepackage{graphics}
\usepackage{amsmath}
\usepackage{amssymb}
\usepackage{comment}
\bibliographystyle{chicago}



\newcommand{\logit}{\mathrm{lgt}}
\newcommand{\I}{\mathrm{I}}
\newcommand{\E}{\mathrm{E}}
\newcommand{\p}{\mathrm{P}}
\newcommand{\e}{\mathrm{e}}
\renewcommand{\o}{\omega}
\newcommand{\vecm}{\mathrm{vec}}
\newcommand{\kp}{\otimes}
\newcommand{\diag}{\mathrm{diag}}
\newcommand{\cov}{\mathrm{cov}}
\newcommand{\eps}{\epsilon}
\newcommand{\ep}{\varepsilon}
\newcommand{\obdots}{\ddots}    % change this later
\newcommand{\Ex}{{\cal E}}
\newcommand{\Exd}{\Ex_d}
\newcommand{\Es}{\E_\phi}
\newcommand{\rat}{{\frac{c_{ij}}{c_{i,j-1}}}}
\newcommand{\rmu}{m}
\newcommand{\rsig}{\nu}
\newcommand{\fd}{\mu}
\newcommand{\tr}{\mathrm{tr}}
\newcommand{\cor}{\mathrm{cor}}
\newcommand{\bx}[1]{\ensuremath{\overline{#1}|}}
\newcommand{\an}[1]{\ensuremath{a_{\bx{#1}}}}

\newcommand{\bi}{\begin{itemize}}
\newcommand{\ei}{\end{itemize}}

\renewcommand{\i}{\item}
\newcommand{\sr}{\ensuremath{\mathrm{SRISK}}}
\newcommand{\cs}{\ensuremath{\mathrm{CS}}}
\newcommand{\cri}{\ensuremath{\mathrm{Crisis}}}
\newcommand{\var}{\ensuremath{\mathrm{VaR}}}
\newcommand{\covar}{\ensuremath{\mathrm{CoVaR}}}
\newcommand{\med}{\ensuremath{\mathrm{m}}}
\newcommand{\de}{\mathrm{d}}
\renewcommand{\v}{\ensuremath{\mathrm{v}_q}}
\newcommand{\m}{\ensuremath{\mathrm{m}}}
\newcommand{\tvar}{\ensuremath{\mathrm{TVaR}}}
\renewcommand{\c}{\ensuremath{\mathrm{CoVaR_q}}}
\renewcommand{\v}{\ensuremath{\mathrm{VaR}_q}}



\newcommand{\eref}[1]{(\ref{#1})}
\newcommand{\fref}[1]{Figure \ref{#1}}
\newcommand{\sref}[1]{\S\ref{#1}}
\newcommand{\tref}[1]{Table \ref{#1}}
\newcommand{\aref}[1]{Appendix \ref{#1}}




\newcommand{\cq}{\ , \qquad}
\renewcommand{\P}{\mathrm{P}}
\newcommand{\Q}{\mathrm{Q}}

\newcommand{\Vx}{{\cal V}}
\newcommand{\be}[1]{\begin{equation}\label{#1}}
\newcommand{\ee}{\end{equation}}

\begin{document}

\section*{Month to month monitoring  of financial stress}
 \tref{twodates} contains real time stress calculations on the first trading day of  January 2009 and November 2014.  As before there is no look ahead bias -- all calculations on each of the two dates only use data available on the first day of the applicable month.   The initial eight rows in \tref{twodates} lists the eight banks used in this study. 
 
 \begin{table}[ht]
\caption{One month ahead stress calculations for Australian banks$^\dag$}\label{twodates}
\centering
\small
\vspace{4mm}
\begin{tabular}{l|rrrrr|rrrrr}
\hline
&\multicolumn{5}{c|}{January 2009}&\multicolumn{5}{c}{November 2014}\\
  \hline
bank & B-log& debt &\multicolumn{2}{c}{stress}&  B-def   & B-log& debt   & \multicolumn{2}{c}{stress}& B-def \\
  \cline{4-5}\cline{9-10}
       &  lev& prop    & back & syst & prob & lev& prop  & back & syst & prob \\
  \cline{2-11} 
         & $\ell_{it} $ & $\pi_{it}$ & $\mu^*_{it}$  & $s^*_{it}$ & $q_{it}$  
         & $\ell_{it} $ & $\pi_{it}$ & $\mu^*_{it}$  & $s^*_{it}$ & $q_{it}$  
        \\
  \hline
cba & 18.57 & 24.30 & 22.99 & 29.60 & 91.43 & -70.34 & 22.70 & 0.00 & 0.00 & 0.00 \\ 
  anz & 15.93 & 18.37 & 14.87 & 18.89 & 92.82 & -38.64 & 22.10 & 0.00 & 0.00 & 0.00 \\ 
  nab & 31.97 & 25.50 & 39.98 & 19.41 & 99.97 & -12.45 & 25.61 & 50.52 & 67.13 & 1.23 \\ 
  wbc & -1.55 & 22.84 & 5.86 & 23.94 & 40.42 & -53.84 & 22.03 & 0.00 & 0.00 & 0.00 \\ 
  mqg & 41.22 & 5.75 & 11.02 & 5.46 & 99.92 & -46.10 & 4.33 & 0.17 & 0.12 & 0.03 \\ 
  boq & 63.30 & 1.32 & 3.55 & 0.99 & 99.95 & -17.05 & 1.33 & 0.80 & 1.34 & 0.33 \\ 
  ben & 18.91 & 1.81 & 1.72 & 1.71 & 90.53 & -7.63 & 1.84 & 25.43 & 30.71 & 8.47 \\ 
  aba & -8.12 & 0.10 & 0.00 & 0.00 & 9.97 & 9.22 & 0.07 & 23.07 & 0.71 & 89.63 \\ 
  \hline
  $\Ex_d$ & 18.78 &  & 17.17 & 8.63 & 82.72 & -41.91 &  & 0.03 & 0.12 & 0.54 \\ 
  $\Sigma$ & 17.81 & $^\dag$2.42 & 15.28 & 10.18 & 96.72 & -44.22 & $^\dag$3.27 & 0.00 & 0.00 & 0.00 \\ 
\hline
\end{tabular}
$^\dag$All numbers multiplied by 100 except total debt (in \$bln).
\end{table}
\normalsize

January 2009 was a time of great stress for all eight banks.     The first  and second columns in the two halves of the table body contain the 
 Basel log--leverage $\ell_{it}$ and debt $\pi_{it}\equiv d_{it}/d_t$ (as a proportion of total debt) for each of the banks.  Six of the eight banks were in Basel default with positive capital shortfalls as indicated in by the $\ell_{it}$ column:   the two banks not in Basel default were wbc and aba.    Background and systemic stress
$$
\mu^*_{it} \equiv \frac{\pi_{it}\mu_{it}}{\Ex_d(\mu_{it})}\cq s^*_{it} \equiv \frac{\pi_{it}s_{it}}{\Ex_d(\mu_{it})}\ , 
$$
for the eight banks as well as the probability of a Basel default in one month $q_{it}$ and the proportion $\pi_{it}$ of total debt are displayed in the next 3 columns.    The background stress column indicates most of the background stress arises from nab -- almost 40\% of the total.   The next most background stressed bank is cba with wbc also a substantial contributor.   The mqg bank contributes almost double to background stress compared to the proportion of total debt  it carries.  The other three small banks contribute relatively little to background stress with boq  almost 3 times expected on the basis of its debt.   Systemic stress is highest for the cba, higher than expected on the basis of its debt load and hence cba was most susceptible to stress from additional general market equity devaluation.    All other banks appear have systemic stress comparable to their size in terms of debt load with only nab being less systemically important.    This should be compared to nab's high background stress.

Continuing with January 2009, the final two rows indicate the total amount of stress in the system and it's diversifiability.    
The second last  row labelled $\Ex_d$  displays, in order, $\Ex_d(\ell_{it})$, blank, $\overline{\mu}_t\equiv \Ex_d(\mu_{it})$, $\overline{s}_t\equiv\Ex_d(s_{it})$ and  $\Ex_d(q_{it})$.   The final row labelled $\Sigma$ displays the aggregate stress quantities,treating all eight banks as one entity, $\ell_t$, total debt $d_t$ in billions of dollars,  $\mu_t$, $s_t$, and  $q_t$.
On an aggregate basis background stress is about twice systemic stress.   Thus there is more danger of increasing capital shortfall due to market volatility  as opposed to further stress from further substantial general market devaluation.  The final row indicates stresses are not diversifiable:    The marginally  smaller  ``diversified"  background stress is offset by an increase in systemic stress.    Also the diversified probability of Basel default is higher than the debt weighted average.   

Stress readings  alter dramatically when moving to November 2014 -- there is virtually no stress in any bank and the small amount of  stress in the system is diversifiable.    Most of the  background stress is carried by nab with lesser contributions by ben and aba.    All the stress in the minor bank aba is background stress as only nab and nab have substantial systemic stress contributions.   Again, however, it must be emphasised that there is minimal systemic stress in the system.    Only aba has substantial Basel default probability, but this bank is a very minor player in the Australian banking scene.   Notice total debt in the banking sector  jumps about 35\% between January 2009 and November 2014.



\end{document}
